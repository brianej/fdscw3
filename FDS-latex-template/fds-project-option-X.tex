\documentclass[11pt,a4paper]{article}
\usepackage[utf8]{inputenc}
\usepackage[T1]{fontenc}
%\usepackage{gentium}
\usepackage{mathptmx} % Use Times Font

\usepackage{graphicx} % Required for including pictures
\usepackage{hyperref} % Format links for pdf
\usepackage{biblatex}
\addbibresource{references.bib}
\usepackage{booktabs} % Used so that tables generated by pandas
                      % to_latex() work correctly

\frenchspacing % No double spacing between sentences
\usepackage[margin=1in]{geometry}

\usepackage[all]{nowidow} % Tries to remove widows
\usepackage[protrusion=true,expansion=true]{microtype} % Improves typography, load after fontpackage is selected

\usepackage{lipsum} % Used for inserting dummy 'Lorem ipsum' text into the template

\title{Cancellation Rates of Hospitals and Health Boards Across Scotland}
\author{S2291980 and 2347484}

\begin{document}

\maketitle

%% INSTRUCTIONS:
%%
%% 1. Create your own copy of this Overleaf project. You can either edit your report
%% using:
%%
%%    a. Overleaf professional, a collaborative LaTeX editor. You can click
%%       "Copy Project" from the Overleaf menu to create a version where you have
%%       read and write permissions. See the following for documentation:
%%       https://www.overleaf.com/edu/edinburgh and
%%       https://uoe.sharepoint.com/:f:/r/sites/digitalskillsandtraining/Shared%20Documents/LaTeX/LaTeX%20for%20Beginners%20using%20Overleaf?csf=1&web=1&e=cPqTI3
%%
%%    b. A LaTeX editor on your PC. For this option, you can download the source
%%       of this project as a zip (via the Overleaf menu).
%% 
%% 2. Please rename this file fds-project-option-1.tex, 
%% fds-project-option-2.tex, or fds-project-option-3.tex, depending on
%% which project option you are doing. When you submit, please submit
%% the PDF file with the corresponding name.
%% 
%% 3. Please keep the section and paragraph headings as they
%%    are. You should delete all the text within the headings, e.g.
%%    the text that says "What is the area of this data science
%%    study, and why is it interesting to investigate" and the
%%    bullet points. Keeping the headings makes the report a lot
%%    easier for the markers to read, and making things easy for
%%    markers is always beneficial.
%%
%% 4. The word limit for the Overview section is mandatory. For the
%% other sections word limits are suggested.
%%
%% 5. The page limits must be strictly adhered to, and depend on if
%% you are working individually, in pairs or in threes:
%%
%%   - Individual: 6 pages 
%%   - Pairs: 8 pages 
%%   - Threes: 10 pages 
%%

\section{Overview}
% 250 words maximum



\section{Introduction}
% Suggested 400 words

\paragraph{Context and motivation}

This study will analyse the cancellation rate of planned operations across hospitals and health boards across Scotland. As cancellations of operations could be caused by various reasons, we are trying to dive deeper into trends and causes of reasons behind the cancellation.


\paragraph{Previous work}

Cancelled operations: a 7-day cohort study of planned adult inpatient surgery in 245 UK National Health Service hospitals \cite{wong_harris_moonesinghe_2018}, a paper that looked through reasons for cancellation of surgery in the UK over a 7-day period across 245 NHS hospitals in March 2017.

Assessing the Rates and Reasons of Elective Surgical Cancellations on the Day of Surgery: A Multicentre Study from Urban Indian Hospitals \cite{sarang_bhandoria_patil_gadgil_bains_khajanchi_kizhakkeveetil_dutta_shah_bhandarkar_etal._2021} examines the frequency and causes of elective surgical cancellations in 10 hospitals across India. They found that most of the cancellations were avoidable and more preparation could decrease the cancellation rate to a more acceptable rate. 

Reasons for cancellation of elective operations at a major teaching referral hospital in Jordan \cite{mesmar_shatnawi_faori_khader_2011}, studied the rate and reasons for cancellations of scheduled operations in King Abdullah University Hospital between August 2005 and July 2006.


\paragraph{Objectives}
We are setting out to answer the following:

\begin{itemize}
    \item Explain fluctuations in cancelled operations over time, especially before and after COVID-19.
    \item Investigate seasonal variations in non-clinical cancellations.
\end{itemize}



\section{Data}
% Suggested 300 words

\paragraph{Data provenance} 
The data used in this project was created by the PHS Waiting Times Team and was aggregated by the NHS Board. It is easily accessible on the Scottish Health and Social Care Open Data website \cite{public_health_scotland}. We downloaded the data in CSV format and were able to use it for our project because it falls under the Open Government License \cite{the_national_archives_2019}. We can adapt and publish the information as long as we give credit to the original source.


\paragraph{Data description} 
There were 3 datasets downloaded from the website, Cancellations by Hospital, Cancellations by Health Board, and Cancellations in Scotland. 

For Cancellations by Hospital, there are 4502 entries and 14 columns: Month (Month and Year of the entry (YYYYMM)), Hospital (Unique hospital code for a NHS Hospital), TotalOperations (Total number of scheduled elective operations), TotalOperationsQF, TotalCancelled (Total number of cancelled operations), TotalCancelledQF, CancelledByPatientReason (Total number of cancelled operations by patients), CancelledByPatientReasonQF, ClinicalReason (Total number of cancelled operations due to clinical reasons), ClinicalReasonQF, NonClinicalCapacityReason (Number of operations cancelled due to non-clinical reasons), and NonClinicalCapacityReasonQF.

For Cancellations by Health Board, there are 1575 entries and 14 columns: Mont h(Month and Year of the entry (YYYYMM)), HBT (Unique code to identify each Health Board), TotalOperations (Total number of scheduled elective operations), TotalOperationsQF, TotalCancelled (Total number of cancelled operations), TotalCancelledQF, CancelledByPatientReason (Total number of cancelled operations by patients), CancelledByPatientReasonQF, ClinicalReason (Total number of cancelled operations due to clinical reasons), ClinicalReasonQF, NonClinicalCapacityReason (Number of operations cancelled due to non-clinical reasons), and NonClinicalCapacityReasonQF.

For Cancellations in Scotland, there are 105 entries and 14 columns: Month (Month and Year of the entry (YYYYMM)), Country (9-digit code for country of treatment), TotalOperations (Total number of scheduled elective operations), TotalOperationsQF, TotalCancelled (Total number of cancelled operations), TotalCancelledQF, CancelledByPatientReason (Total number of cancelled operations by patients), CancelledByPatientReasonQF, ClinicalReaso n(Total number of cancelled operations due to clinical reasons), ClinicalReasonQF, NonClinicalCapacityReason (Number of operations cancelled due to non-clinical reasons), and NonClinicalCapacityReasonQF.


\paragraph{Data processing} 
Some cleaning was required for the three datasets. About half of the columns in each dataset were useless and needed to be dropped. These were columns that ended with a QF. In the Cancellations by Health Board dataset, one entry was missing, and there was a month when a Health Board entry had no value. We collected all the health boards for that particular month and concluded that the missing entry belonged to one of the health boards that also did not have an entry for that specific month.

According to the website, there appears to be a discrepancy in the data obtained from NHS Orkney, and efforts are being made to resolve the issue. The statistics are subject only to basic quality assurance by PHS. As a result, these datasets may not present a complete overview of cancellations throughout Scotland, and the data should not be trusted entirely.

After conducting a comprehensive review of the datasets, we discovered that the Cancellations by Hospitals dataset contains information for only 49 hospitals, which is only half of the total NHS hospitals in Scotland \cite{hospital_scotland}. Out of these 49 hospitals, only 30 have complete data available from May 2015 until January 2024.


\section{Exploration and  analysis}
% Suggested 500 words for individual report; proportionately longer
% for group projects).



A data science analysis of the paper, including: 
\begin{itemize}
\item Visualisations (for example
  Figure~\ref{fds-project-template:fig:example1}) and tables (for
  example Table~\ref{tab:example1}). Please make sure that all figures
  and tables are referred to in the text, as demonstrated in this
  bullet point. 
\item Interpretation of the results 
\item Description of how you have applied one ore more of the
  statistical and ML methods learned in the FDS to the data
\item Interpretation of the findings 
\end{itemize}

You can use equations like this:
\begin{equation}
  \label{fds-project-template:eq:1}
  \overline{x} = \sum_{i=1}^n x_i
\end{equation}
or maths inline: $E=mc^2$. However, you do not need to reexplain techniques that you have learned in the course -- assume the reader understands linear regression, logistic regression K-nearest neighbours etc.  Remember to explain any symbols use, e.g.~``$n$ is the number of data points and $x_i$ is the value of the $i$th data point.''.

\section{Discussion and conclusions}
% Suggested 400 words.

\paragraph{Summary of findings}

\paragraph{Evaluation of own work: strengths and limitations}

\paragraph{Comparison with any other related work}
E.g. ``Anscombe has also demonstrated that many patterns of data can
have the same correlation coefficient'' \cite{anscombe1973graphs}.

Wikipedia can also be cited but it is better if you find the original
reference it for a particular claim in the list of references on the
Wikipedia page, read it, and cite it.

The golden rule is always to cite information that has come from other
sources, to avoid plagiarism \cite{wiki:plagarism}.

\paragraph{Improvements and extensions}


\printbibliography
\end{document}

% LocalWords:  lrrrrrrr ment Macduff Kemnay Ruchill FDS mc th fds
% LocalWords:  Anscombe